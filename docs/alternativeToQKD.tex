\title{A commercial alternative to QKD: filling USB sticks and hard disks with random bits}

\documentclass{paper}

\usepackage{geometry}
\geometry{
	a4paper,	
	margin=20mm
}

\usepackage{hyperref}

\begin{document}
	\maketitle
	\section{Introduction}
		QKD (Quantum Key Cryptography) offers total security in accordance with the laws of quantum physics, but is limited by its expense and limited range. QKD requires the initial distribution of small keys to start the system running, this is usually done by physical means. This begs the question, is it more commercially viable to distribute disks filled with large amounts of random data for use in an encryption system?\\
		This approach would allow symmetric key cryptography in the same way it is facilitated by QKD, but also allows for a more flexible system as the vast number of bits are immediately available, allowing a one time pad to be used.\\
		
	
	\section{Randomness Sources}
		To build a one time pad encryption system large amounts of randomly generated bits are needed. These random bits should not be reproducible as it would render the system useless. Therefore suitable sources of cryptographically secure random numbers must be used. These sources must also have a high throughput due meet the needs of large amounts of bits.\\ 
	
		\subsection{Intel RDRAND}
			Intel RDRAND has an incredibly high maximum throughput at 800 MB/sec. (Note though that a single thread will see a throughput of 70 to 200MB/sec).\footnote{\url{https://software.intel.com/sites/default/files/managed/4d/91/DRNG_Software_Implementation_Guide_2.0.pdf}} \\
			My own tests show that we could fill a 1TB disk in 6 Hours, 09 Minutes and 32 Seconds using RDRAND at throughput rates of:\\
			0.022365s = 44.71 MB/s\\
			0.023165s = 43.17 MB/s\\
			0.021080s = 47.44 MB/s\\
			Mean throughput of 45.1 MB/s\\
			This throughput rate is more than suitable for our needs.\\\\
			Intel RDRAND has been found to be robustly designed and is unlikely to provide insecure predictable randomness, this is mostly due to the usage of a von Neumann corrector. ``the most likely failure modes cause the output to be “stuck” in one state (e.g., stuck on), causing no output from the von Neumann corrector'' \footnote{Jun, B. and Kocher, P.``The Intel Random Number Generator'' Cryptography Research, Inc., April 1999}

			There is much talk about possible backdoors and trojans into Intels RDRAND system, it has been found that it is possible to insert a 
			Trojan into the Intel CPU without detection. 
			``Since optical reverse-engineering is not feasible and our Trojan passes functional testing, a verifier cannot distinguish a Trojan design from a Trojan-free design''\footnote{Becker, Georg T., et al.``Stealthy dopant-level hardware trojans.'' International Workshop on Cryptographic Hardware and Embedded Systems. Springer Berlin Heidelberg, 2013}
				
			Intel's David Johnston states ``I’ve examined my own RNG with electron microscopes and picoprobes. So I and a number of test engineers know full well that the design hasn’t been subverted.''\footnote{\url{https://plus.google.com/+TheodoreTso/posts/SDcoemc9V3J}}
				
			Due to the risk of an undetected hardware trojan it's preferable (but not imperative) that the Intel TRNG is not the sole source of entropy in the system.
			
		\subsection{ID Quantique Quantis}
			ID Quantique Quantis-USB throughput of 4 Mbit/s this is 500kB/s, the same speed which we get from Linux insecure /dev/urandom.\\
			Combining this with Intels TRNG or using it as a sole source of entropy would enable us to generate random bits at a rate of 1.8GB/hour. \\
			ID Quantique do offer a 16Mbit/s PCIe version of the card which would enable us to generate bits at a rate of 7.2GB/hour.\footnote{\url{http://marketing.idquantique.com/acton/attachment/11868/f-004b/1/-/-/-/-/Quantum\%20RNG\%20White\%20Paper.pdf}}
			
			
		\subsection{Linux /dev/random}
			There is little analysis into the Linux implementation of of /dev/random and its PRNG as a whole.\\
			It involves 3 entropy pools, one primary which feeds the secondary /dev/random and /dev/urandom pools, the pools are filled by keyboard use, mouse use, disk use, interrupts and hardware entropy such as Intel RDRAND if possible. 
			These will then by hashed together using SHA-1 to provide the output bits.\footnote{Gutterman, Zvi, Benny Pinkas, and Tzachy Reinman. ``Analysis of the linux random number generator.'' Security and Privacy, 2006 IEEE Symposium on. IEEE, 2006.} \\
			It is accepted that this approach is cryptographically secure, but very slow to generate bits with my tests showing a mean throughput of 26.5kB/s.  Due to this is not useful to consider it for use in the system.
			
		\subsection{CSPRNGs}
			Each of the mentioned ways of generating random bits rely on TRNGs (True Random Number Generators).  CSPRNG (Cryptographically Secure Psuedo Random Number Generators) exist (such as using AES in CTR mode) but are difficult for us to use due to the fact they need 
			to be seeded by another source of cryptographically secure random data.\\ 
			We could do this by seeding it with random data supplied by the Intel instruction RDSEED (Although, we have no access to hardware which can complete this instruction). Or by data from RDRAND directly. These approaches are not sufficient though as we are seeking to use a CSPRNG in conjunction with Intel RDRAND. Seeding with Intel's RDRAND would mean that the Intel TRNG would remain 	the sole root source of all random data.\\
			A better alternative would be to seed from /dev/random due to making the most its slow speed and accepted security.\\
			
			A CSPRNG can be built by using AES in CTR mode. I found that using OpenSSL with AES-NI instructions resulted in a CSPRNG which was incredibly fast. It is seeded with data from /dev/random and yields throughput of 29.281 MB/s. This would enable us to fill a 1TB disk in 9 Hours, 26 Minutes and 35 Seconds. 
			
		\subsection{Statistical Testing}
			The suite of statistical tests used on the random bits we generated is the DieHarder Suite.\\ 
			DieHarder includes the DieHard set of statistical tests along with tests from NIST's (National Institute of Standards and Technology) STS (Statistical Test Suite).\\ 
			RDRAND and the AES CSPRNG were tested due to their acceptable speeds making them good candidates for the system. Both of these test results show a small amount of `WEAK' results from the analysis. These detailed results can be found on pastebin.\footnote{AES CSPRNG:\url{https://pastebin.com/raw/gcH4jXGp} RDRAND:\url{https://pastebin.com/raw/diEgthVN}}\\
			This can also be explained by the DieHarder manual pages which state ``If you run a long test series, you will see occasional weak returns for a perfect generators because p is uniformly distributed and will appear in any finite interval from time to time.''\footnote{\url{http://manpages.ubuntu.com/manpages/precise/man1/dieharder.1.html}}
			
		\subsection{Chosen Source}
			Combining Intel RDRAND with an AES CSPRNG through XOR was chosen as the source of random bits in our system. This is due the speed of both sources and the potential security risk of RDRAND meaning it is preferable not to be used alone. The two solutions also scored well on the DieHarder tests used to verify their randomness.

	\newpage
		
		\section{File Layout}
						
\end{document}