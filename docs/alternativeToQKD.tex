\title{A commercial alternative to QKD: filling USB sticks and hard disks with random bits}

\documentclass{paper}

\usepackage{geometry}
\geometry{
	a4paper,	
	margin=20mm
}


\begin{document}
	\maketitle
	\section{Introduction}
		QKD (Quantum Key Cryptography) offers total security in accordance with the laws of quantum physics, but is limited by its expense and limited range. QKD requires the initial distribution of small keys to start the system running, this is usually done by physical means. This begs the question, is it more commercially viable to distribute disks filled with large amounts of random data for use in a one time pad system? \\
		
	
	\section{Randomness Sources}
			
						
\end{document}
